% 
% Permission is granted to copy, distribute and/or modify this document
% under the terms of the GNU Free Documentation License, Version 1.2
% or any later version published by the Free Software Foundation;
% with no Invariant Sections, no Front-Cover Texts, and no Back-Cover
% Texts.  A copy of the license is included in the section entitled "GNU
% Free Documentation License".




%%%%%%%%%%%%%%%%%%%%%%%%%%%%%%%%%%%%%%%%%%%%%%%%%%%%%%%%%%%%%%%%%%%%%%%%%%%%%%%%%%%%%%%%%% 
\section{Use Cases Guide}

This section presents the main functionalities of the module \textit{otpmml} in their context.



%%%%%%%%%%%%%%%%%%%%%%%%%%%%%%%%%%%%%%%%%%%%%%%%%%%%%%%%%%%%%%
\subsection{NeuralNetwork import}

Class \texttt{NeuralNetwork} imports a neural network model from a \texttt{PMML} file. 


Python script for this use case :

\begin{lstlisting}
import openturns as ot
from otpmml import NeuralNetwork

# Import the neural network
neural_network = NeuralNetwork("myPMMLFile.pmml")
print (neural_network)

\end{lstlisting}


%%%%%%%%%%%%%%%%%%%%%%%%%%%%%%%%%%%%%%%%%%%%%%%%%%%%%%%%%%%%%%
\subsection{RegressionModel import}

Class \texttt{RegressionModel} imports a regression model from a \texttt{PMML} file.


Python script for this use case :

\begin{lstlisting}
import openturns as ot
from otpmml import RegressionModel

# Import the model
model = RegressionModel("myPMMLFile.pmml")
print (model)

# LinearLeastSquares accessor
linearLeastSquares = model.getLinearLeastSquares()

# Export model to Pmml file
model.exportToPMMLFile("linearModel.pmml")

\end{lstlisting}

%%%%%%%%%%%%%%%%%%%%%%%%%%%%%%%%%%%%%%%%%%%%%%%%%%%%%%%%%%%%%%
\subsection{Import and export of DAT files}

Class \texttt{IO} exports both input\slash output samples into a \texttt{DAT} file


Python script for this use case :

\begin{lstlisting}
import openturns as ot
import otpmml.DAT as DAT
from math import pi, sin

a = 7.0
b = 0.1
# Create the Ishigami function
input_variables = ["xi1","xi2","xi3"]
formula = ["sin(xi1) + (" + str(a) + ") * (sin(xi2)) ^ 2 + (" + str(b) + ") * xi3^4 * sin(xi1)"]
model = ot.NumericalMathFunction(input_variables, formula)
model.setName("Ishigami")
# Generating
dist = ot.ComposedDistribution(3 *[ot.Uniform(-pi,pi)])
X = dist.getSample(100)
Y = model(X)
# Export to DAT file
DAT.Export("data.dat",X, Y)

# Import DAT file
sampleCollection = DAT.Import("data.dat")
input_sample = sampleCollection[0]
output_sample = sampleCollection[1]
\end{lstlisting}

Export could be done also using only one sample.


Python script for this use case :

\begin{lstlisting}
import openturns as ot
import otpmml.DAT as DAT
from math import pi, sin

a = 7.0
b = 0.1
# Create the Ishigami function
input_variables = ["xi1","xi2","xi3"]
formula = ["sin(xi1) + (" + str(a) + ") * (sin(xi2)) ^ 2 + (" + str(b) + ") * xi3^4 * sin(xi1)"]
model = ot.NumericalMathFunction(input_variables, formula)
model.setName("Ishigami")
# Generating
dist = ot.ComposedDistribution(3 *[ot.Uniform(-pi,pi)])
X = dist.getSample(100)
Y = model(X)
# Encapsulating both samples
Z = ot.NumericalSample(X)
Z.stack(Y)
# Export to DAT file
DAT.Export("data.dat",Y)
\end{lstlisting}
