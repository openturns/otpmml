% 
% Permission is granted to copy, distribute and/or modify this document
% under the terms of the GNU Free Documentation License, Version 1.2
% or any later version published by the Free Software Foundation;
% with no Invariant Sections, no Front-Cover Texts, and no Back-Cover
% Texts.  A copy of the license is included in the section entitled "GNU
% Free Documentation License".

%%%%%%%%%%%%%%%%%%%%%%%%%%%%%%%%%%%%%%%%%%%%%%%%%%%%%%%%%%%%%%%%%%%%%%%%%%%%%%%%%%%%%%%%%% 
\section{User Manual}

This section gives an exhaustive presentation of the objects and functions provided by the \textit{otpmml} module, in the alphabetic order.


\subsection{DAT}

This class is used through its static methods in order to import\slash export a \textit{NumericalSample} or a collection of \textit{NumericalSample} into a \texttt{.dat} file.

\begin{description}

\item[Methods:]  \rule{0pt}{1em}

\begin{description}
\item \textit{Import}
\begin{description}
\item[Usage:] \rule{0pt}{1em}
\begin{description}
\item  \textit{DAT.Import(filename)}
\end{description}
\item[Arguments:] \rule{0pt}{1em}
\begin{description}
\item \textit{filename}: a string, file that contains data
\end{description}
\item[Value:]  a collection of samples of size 2. First sample corresponds to input data, second one to output data.
\end{description}
\end{description}
\bigskip

\begin{description}
\item \textit{Export}
\begin{description}
\item[Usage:] \rule{0pt}{1em}
\begin{description}
\item  \textit{DAT.Export(filename, input, output)}
\item  \textit{DAT.Export(filename, inputOutput)}
\end{description}
\item[Arguments:] \rule{0pt}{1em}
\begin{description}
\item \textit{filename}: a string, file where to export data.
\item \textit{input}: a \textit{NumericalSample}, input sample, usually of dimension $\geq 1$.
\item \textit{output}: a \textit{NumericalSample}, output sample, usually of dimension $1$.
\item \textit{inputOutput}: a \textit{NumericalSample}, usually of dimension $\geq 1$.
\end{description}
\item[Value:]  None.
\end{description}
\end{description}
\bigskip
\end{description}

\subsection{NeuralNetwork}

The class inherits from the \textit{NumericalMathFunction} class.

\begin{description}

\item[Usage:] \rule{0pt}{1em}
  \begin{description}
  \item \textit{NeuralNetwork(pmmlFile)}
  \end{description}

\item[Arguments:]  \rule{0pt}{1em}
  \begin{description}
  \item \textit{pmmlFile}: a string, PMML file that countains the neural network
  \end{description}

\item[Value:] a NeuralNetwork, a \textit{NumericalMathFunction} that implements neural network

\item[Details:]  \rule{0pt}{1em}
  \begin{description}
  \item NeuralNetwork constructor
  \end{description}

\item[Links] \rule{0pt}{1em}
\end{description}

\subsection{RegressionModel}

\begin{description}

\item[Usage:] \rule{0pt}{1em}
  \begin{description}
  \item \textit{RegressionModel(pmmlFile)}
  \item \textit{RegressionModel(linearLeastSquares)}
  \end{description}

\item[Arguments:]  \rule{0pt}{1em}
  \begin{description}
  \item \textit{pmmlFile}: a string, PMML file that countains the regression model
  \item \textit{linearLeastSquares}: a LinearLeastSquare, object encapsulating least squares.
  \end{description}

\item[Value:] a RegressionModel

\item[Details:]  \rule{0pt}{1em}
  \begin{description}
  \item With the first usage, the class loads a model implemented in \texttt{PMML} format
  \item With the second usage, the class encapsulates a LinearLeastSquares attribut
  \end{description}

\item \textit{getLinearLeastSquare}
    \begin{description}
    \item[Usage:] \textit{getLinearLeastSquare()}
    \item[Arguments:] no argument
    \item[Value:] a \textit{LinearLeastSquare}
    \end{description}
    \bigskip
    
\item \textit{exportToPMMLFile}
    \begin{description}
    \item[Usage:] \textit{exportToPMMLFile(filename)}
    \item[Arguments:] \textit{filename}, a string. Name of file for the export of regression model.
    \item[Value:] none.
    \end{description}
    \bigskip

% \item[Links] \rule{0pt}{1em}
\end{description}
